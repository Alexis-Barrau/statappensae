\documentclass[a4paper, 12pt]{article}
\usepackage[utf8]{inputenc}
\usepackage[T1]{fontenc}

\begin{document}
\section{Introduction}
Le vote constitue un problème classique en sciences sociales, depuis les travaux pionniers d’André Siegfried sur la géographie électorale. Les analyses quantitatives ultérieures, inaugurées aux Etats-Unis par Paul Lazarsfeld et son équipe dans les années 1940, puis développées en France à partir des années 1970, ont isolé plusieurs grands déterminants du vote : l’appartenance socioprofessionnelle et religion d’abord, puis l’âge, le sexe, le degré d’instruction, le lieu de résidence, etc. Schématiquement, les catholiques et les indépendants votent à droite, les salariés (dont particulièrement les ouvriers et les fonctionnaires) votent à gauche, tandis que les jeunes, les chômeurs et les peu diplômés s’abstiennent (Douillet 2023). Les effets de ces variables sont toutefois interactifs, évoluent parfois au cours du temps, et sont conditionnés par l’offre électorale, ce qui s’oppose à une lecture trop mécaniste des pratiques de vote.
\section{Méthodes}
\subsection{Une première sous-section}
Le texte ne commence pas par un alinéa.
\end{document}

